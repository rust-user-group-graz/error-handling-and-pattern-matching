\documentclass{beamer}
\usetheme{metropolis}
%\setsansfont[BoldFont={Fira Sans SemiBold}]{Fira Sans Book}
%\setsansfont{Fontin}
%\setsansfont{Gillius ADF No2}
%\setsansfont{Phetsarath OT}
\setsansfont{Source Sans Pro}
\setmonofont{Source Code Pro}

\hypersetup{colorlinks=true,
            linkcolor=mRustLightOrange,
            menucolor=mRustLightOrange,
            pagecolor=mRustLightOrange,
            urlcolor=mRustLightOrange}
\usepackage{csquotes}
\usepackage{comment}
\usepackage{xcolor}
\usepackage{minted}

\newfontfamily\codefont{Source Code Pro}
\newcommand\code[1]{\,{\color[HTML]{884400}#1}\,}
\newcommand\source[1]{$\rightarrow$ via #1}

\title{traits \& generics}
\date{\today}
\author{Lukas Prokop}
\institute{RustGraz community\vfill\hfill\includegraphics[height=2cm]{images/rustacean-orig-noshadow.png}}
\begin{document}
\maketitle

% TODO Feedback(David): turn hyperlinks into footnotes

\section{Prologue}
\begin{frame}[fragile]{std::mem::size\_of\_val of union}
  \textbf{Last session:} One \code{unsafe} superpower is \enquote{Access fields of unions}. \\
  \textbf{Claim:} The size of a union is the size of its largest member.

  \begin{minted}{rust}
  pub const fn size_of<T>() -> usize
  \end{minted}

  Returns the size of a type in bytes.

  More specifically, this is the offset in bytes between successive elements in an array with that item type including alignment padding. Thus, for any type \mintinline{rust}{T} and length \mintinline{rust}{n}, \mintinline{rust}{[T; n]} has a size of \mintinline{rust}{n * size_of::<T>()}.

  In general, the size of a type is not stable across compilations, but specific types such as primitives are.

  \source{\href{https://doc.rust-lang.org/std/mem/fn.size_of.html}{rust doc}}
\end{frame}

\begin{frame}[fragile]{std::mem::size\_of\_val of union}
  \textbf{Reminder:} a slide from December:

  \begin{minted}{rust}
use std::mem;

fn main() {
  println!("{}", mem::size_of::<A>());
}
\end{minted}

  where \mintinline{rust}{A} is \mintinline{rust}{u8} (1), \mintinline{rust}{u32} (4), \mintinline{rust}{f64} (8), \mintinline{rust}{&u8} (8), \mintinline{rust}{String} (24), \mintinline{rust}{&str} (16), \mintinline{rust}{Vec<u8>} (24) or \mintinline{rust}{&[char]} (16).
\end{frame}

\begin{frame}[fragile]{std::mem::size\_of\_val of union}
  \textbf{Size of} \mintinline{rust}{#[repr(C)]} \textbf{ items}

  The C representation for items has a defined layout. With this layout, the size of items is also stable as long as all fields have a stable size.

  \textbf{Size of Unions}

  The size of a union is the size of its largest field.

  Unlike C, zero sized unions are not rounded up to one byte in size.

  \source{\href{https://doc.rust-lang.org/std/mem/fn.size_of.html\#size-of-reprc-items}{rust doc}}
\end{frame}

\begin{frame}[fragile]{std::mem::size\_of\_val of union}
  \begin{minted}{rust}
use std::mem::size_of;

#[repr(C)]
union Ambiguous {
  my_int: u32,
  addr: *const u32,
}

fn main() {
  println!("{}", size_of::<Ambiguous>());
  // ⇒ 8
}
\end{minted}
\end{frame}

\begin{frame}[fragile]{alignment of struct}
  \begin{minted}{rust}
  pub const fn align_of<T>() -> usize
  \end{minted}

  Returns the ABI-required minimum alignment of a type.

  Every reference to a value of the type T must be a multiple of this number.

  This is the alignment used for struct fields. It may be smaller than the preferred alignment.

  \source{\href{https://doc.rust-lang.org/std/mem/fn.size_of.html}{rust doc}}
\end{frame}

\begin{frame}[fragile]{alignment of struct}
  \textbf{Size of Structs}

  For structs, the size is determined by the following algorithm.

  For each field in the struct ordered by declaration order:
  \begin{enumerate}
    \item Add the size of the field.
    \item Round up the current size to the nearest multiple of the next field's alignment.
  \end{enumerate}

  Finally, round the size of the struct to the nearest multiple of its alignment. The alignment of the struct is usually the largest alignment of all its fields; this can be changed with the use of \code{repr(align(N))}.

  Unlike C, zero sized structs are not rounded up to one byte in size.
\end{frame}

\begin{frame}[fragile]{alignment of struct}
  \begin{minted}{rust}
use std::mem::{size_of,align_of};

fn main() {
  println!("{} ⇒ {}", size_of::<i32>(),
           align_of::<i32>());       // 4 ⇒ 4
  println!("{} ⇒ {}", size_of::<u8>(),
           align_of::<u8>());        // 1 ⇒ 1
  println!("{} ⇒ {}", size_of::<*mut i32>(),
           align_of::<*mut i32>());  // 8 ⇒ 8
  println!("{} ⇒ {}", size_of::<usize>(),
           align_of::<usize>());     // 8 ⇒ 8
  println!("{} ⇒ {}", size_of::<char>(),
           align_of::<char>());      // 4 ⇒ 4
}
  \end{minted}
\end{frame}

\begin{frame}[fragile]{alignment of struct}
  \begin{minted}{rust}
struct Data {
  a: i32,
  b: u8,
  c: *mut i32,
  d: usize,
  e: char,
}

struct SomeU8 {
  a: u8,
  b: u8,
}
  \end{minted}
\end{frame}

\begin{frame}[fragile]{alignment of struct}
  \begin{minted}{rust}
struct SomeI32 {
  a: i32,
  b: i32,
}

fn main() {
  println!("{} ⇒ {}", size_of::<Data>(),
           align_of::<Data>());    // 32 ⇒ 8
  println!("{} ⇒ {}", size_of::<SomeU8>(),
           align_of::<SomeU8>());  // 2 ⇒ 1
  println!("{} ⇒ {}", size_of::<SomeI32>(),
           align_of::<SomeI32>()); // 8 ⇒ 4
}
  \end{minted}
\end{frame}

\section{Dialogue}

\begin{frame}[fragile]{Goal for today}
  \begin{itemize}
    \item Understanding how traits work
    \item Understanding how generics work
    \item Understanding a few type-system related properties/patterns
    \item Understanding how rust models usecases covered by OOP, FP, concepts, contracts, interfaces, … $\Rightarrow$ discuss type systems
  \end{itemize}
\end{frame}

% POSTPONED: pub?
% use {crate}::{package}::{trait}; ⇒ can implement it on their own. {trait} must be pub.
% public / private / protected?
% pub for individual struct members? Rust 1.41.0 release ⇒ https://github.com/rust-lang/rust/pull/66183/

\section{Dialogue: traits}

\begin{frame}[fragile]{struct}
  \begin{minted}{rust}
struct Talk {
    desc: String,
    duration: u16,
}
  \end{minted}
\end{frame}

\begin{frame}[fragile]{struct}
  \begin{minted}{rust}
impl Talk {
  fn new(text: &str) -> Talk {
    Talk {
      desc: text.to_string(),
      duration: 45,
    }
  }
}
  \end{minted}
  A method \code{new} implemented over \code{struct} type \code{Talk}.
\end{frame}

\begin{frame}[fragile]{struct - \emph{not} like this!}
  \begin{minted}{rust}
fn Talk::new() -> Talk {
    Talk { desc: "foo", duration: 42 }
}
  \end{minted}
  \begin{minted}{text}
error: expected one of `(` or `<`, found `::`
  \end{minted}
  \begin{minted}{rust}
impl Talk {
    fn new() -> Talk {
        Talk { /*…*/ }
    }
    fn new() -> Talk {
        Talk { /*…*/ }
    }
}
  \end{minted}
  \begin{minted}{text}
error: duplicate definitions with name `new`
  \end{minted}
\end{frame}

\begin{frame}[fragile]{traits}
  \begin{minted}{rust}
trait Submission {
    fn len(&self) -> u32;
    fn summary(&self) -> String;
}
  \end{minted}
\end{frame}

\begin{frame}[fragile]{traits cannot be instantiated}
  \begin{minted}{rust}
fn main() {
    let s: Submission;
}
  \end{minted}

  \begin{minted}[fontsize=\tiny]{text}
warning: trait objects without an explicit `dyn` are deprecated
  --> src/main.rs:16:12
   |
16 |     let s: Submission;
   |            ^^^^^^^^^^ help: use `dyn`: `dyn Submission`
   |
   = note: `#[warn(bare_trait_objects)]` on by default

error[E0277]: the size for values of type `dyn Submission` cannot be known at compilation time
  --> src/main.rs:16:9
   |
16 |     let s: Submission;
   |         ^ doesn't have a size known at compile-time
   |
   = help: the trait `std::marker::Sized` is not implemented for `dyn Submission`
   = note: to learn more, visit <https://doc.rust-lang.org/book/ch19-04-advanced-types.html#dynamically-sized-types-and-the-sized-trait>
   = note: all local variables must have a statically known size
   = help: unsized locals are gated as an unstable feature
   \end{minted}
\end{frame}

\begin{frame}[fragile]{traits}
  \begin{minted}{rust}
impl Submission for Talk {
  fn len(&self) -> u32 {
    self.duration as u32
  }
  fn summary(&self) -> String {
    let dot = self.desc.find('.');
    match dot {
      Some(idx) => {
        let mut s = String::new();
        s.push_str(&self.desc[0..idx]);
        s.push_str(" …");
        s
      },
      None => self.desc.clone(),
} } }
  \end{minted}
\end{frame}

\begin{frame}[fragile]{traits}
  \begin{minted}{rust}
fn main() {
  let t = Talk::new(concat!("Rust ",
    "is a multi-paradigm system ",
    "programming language focused ",
    "on safety, especially safe ",
    "concurrency. Rust is syntactically ",
    "similar to C++, but is designed to ",
    "provide better memory safety while ",
    "maintaining high performance. In ",
    "this talk I want to introduce ",
    "listeners to the Rust programming ",
    "language."));
  println!("{}", t.summary());
}
  \end{minted}
\end{frame}

\begin{frame}[fragile]{traits}
  \begin{itemize}
    \item If the first parameter of a method is \mintinline{rust}{&self} or \mintinline{rust}{&mut self}, an object is required.
    \item \mintinline{rust}{t.summary()} is syntactic sugar for \mintinline{rust}{Talk::summary(t)}
    \item Otherwise, we would call it a \emph{static} method in OOP.
    \item \mintinline{rust}{trait} keyword to declare traits
    \item \mintinline{rust}{impl <trait> for <struct>} for implementation
    \item enums, structs, and unions can implement traits
  \end{itemize}
\end{frame}

\begin{frame}[fragile]{traits coherence}
  \begin{quote}
    One restriction to note with trait implementations is that we can implement a trait on a type only if either the trait or the type is local to our crate.
    This restriction is part of a property of programs called \emph{coherence}, and more specifically the \emph{orphan rule}, so named because the parent type is not present.
  \end{quote}
  Ensures that adding a dependency can't break your code.
\end{frame}

\begin{frame}[fragile]{traits coherence}
  Can we implement trait \mintinline{rust}{fmt::Display} for our custom type? \textbf{Yes.} \\[15pt]
  Can we implement our custom trait \mintinline{rust}{ToHTML} for type \mintinline{rust}{Vec<String>}? \textbf{Yes.} \\[15pt]
  Can we implement trait \mintinline{rust}{fmt::Display} for type \mintinline{rust}{Vec<String>}? \textbf{No.}
\end{frame}

\begin{frame}[fragile]{traits coherence}
  Rust release 1.41.0 (released 2020-01-30) includes \href{https://rust-lang.github.io/rfcs/2451-re-rebalancing-coherence.html}{RFC 2451}.

  \enquote{Stabilize the \texttt{re\_rebalance\_coherence} feature} [\href{https://github.com/rust-lang/rust/pull/65879/}{PR~65879}]

  \begin{quote}
    For better or worse, we allow implementing foreign traits for foreign types. For example, \mintinline{rust}{impl From<Foo> for Vec<i32>} is something any crate can write, even though \mintinline{rust}{From} is a foreign trait, and \mintinline{rust}{Vec} is a foreign type. However, under the current coherence rules, we do not allow \mintinline{rust}{impl<T> From<Foo> for Vec<T>}.
  \end{quote}
\end{frame}

\begin{frame}[fragile]{traits default behavior}
  \begin{minted}{rust}
trait Submission {
  fn len(&self) -> u32;
  fn summary(&self) -> String;
  fn mentions_rust(&self) -> bool {
    true
  }
}
  \end{minted}
  \mintinline{rust}{mentions_rust} provides a default implementation. Can be overidden by any implementation.
  Adding one trait need not break backwards-compatibility ($\rightarrow$ extensibility).

  It isn’t possible to call the default implementation from an overriding implementation of that same method.

  \mintinline{rust}{Submission} cannot have any mutable fields like \mintinline{rust}{title: String}.
\end{frame}

\begin{frame}[fragile]{traits constants}
  \begin{minted}{rust}
trait Submission {
  const MAX_DURATION: u32 = 300;
  fn len(&self) -> u32;
  fn summary(&self) -> String;
  fn mentions_rust(&self) -> bool {
    true
  }
}
  \end{minted}
\end{frame}

\begin{frame}[fragile]{traits constants}
  Example via stackoverflow: \href{https://stackoverflow.com/q/56718406}{Why does Rust assignment of const from trait to trait not work?}
\end{frame}

\begin{frame}[fragile]{traits constants}
  \begin{minted}{rust}
trait A {
    const X: i32 = 1;
}
struct S;
impl A for S {}
trait B {
    const Y: i32 = A::X;
}
trait C {
    const Y: i32 = S::X;
}
fn main() {
}
  \end{minted}
\end{frame}

\begin{frame}[fragile]{traits constants}
  \begin{itemize}
    \item Does not compile.
    \item In essence, any trait cannot depend on values of other traits.
    \item Constants are constants. Not default values (for other types).
  \end{itemize}
\end{frame}

\begin{frame}[fragile]{traits constants}
  \begin{minted}[fontsize=\scriptsize]{rust}
trait A {
    const X: i32 = 1;
}
struct S;
impl A for S {}
struct R;
impl A for R {
    const X: i32 = 42;
}

fn main() {
    println!("S: {}", S::X);        // S: 1
    println!("R: {}", R::X);        // R: 42
    // A::X alone is ambiguous
    println!("S: {}", <S as A>::X); // S: 1
    println!("R: {}", <R as A>::X); // R: 42
}
  \end{minted}
\end{frame}

\begin{frame}[fragile]{traits constants}
  \begin{minted}[fontsize=\scriptsize]{rust}
trait A {
    fn get_x() -> i32 {
        1
    }
}
struct S;
impl A for S {}
struct R;
impl A for R {
    fn get_x() -> i32 {
        42
    }
}

fn main() {
    println!("S: {}", S::get_x());        // S: 1
    println!("R: {}", R::get_x());        // R: 42
    println!("S: {}", <S as A>::get_x()); // S: 1
    println!("R: {}", <R as A>::get_x()); // R: 42
}
  \end{minted}
\end{frame}

\begin{frame}[fragile]{traits default behavior}
  \begin{minted}{rust}
fn submit_to_glt(sub: impl Submission) {
  println!("Dear GLT-Team, we would like \
    to submit the following talk: {}",
    sub.summary());
}
  \end{minted}

  The parameter can be \emph{anything} satisfying the trait \emph{Submission}.
\end{frame}

\begin{frame}[fragile]{traits and trait bounds}
  \begin{minted}{rust}
fn submit_to_glt<T: Submission>(sub: T) {
  println!("Dear GLT-Team, we would like \
    to submit the following talk: {}",
    sub.summary());
}
  \end{minted}

  Another syntax.
  This is called \emph{trait bound}.
\end{frame}

\begin{frame}[fragile]{traits and trait bound unions}
  \begin{minted}{rust}
fn represent_submission
   <T: Submission + Display>
   (sub: T) {
  // …
}
  \end{minted}

  The parameter must implement Submission \emph{and} Display.
\end{frame}

% TODO I never discussed whether/when “impl<T> Trait<T>” or “impl Trait<T>” is correct

\begin{frame}[fragile]{traits and trait bounds with where}
  \begin{minted}{rust}
fn submit_to_glt<T>(sub: T)
    where T: Submission {
  println!("Dear GLT-Team, we would like \
    to submit the following talk: {}",
    sub.summary());
}
  \end{minted}

  More generic example:
  \begin{minted}{rust}
impl <A, D> MyTrait<A, D> for YourType where
    A: TraitB + TraitC,
    D: TraitE + TraitF {}
  \end{minted}
\end{frame}

\begin{frame}[fragile]{traits and trait bounds with bang}
  \begin{minted}{rust}
impl <A, D> MyTrait<A, D> for YourType where
    A: TraitB + TraitC,
    D: !TraitE + TraitF {}
  \end{minted}

  This example does \emph{not} compile. \emph{Negative \texttt{!trait} bounds} were dropped (\href{https://github.com/rust-lang/rfcs/pull/586}{PR 586 \enquote{Negative bounds}}).
\end{frame}

\begin{frame}[fragile]{traits and ?Sized}
  \begin{minted}{rust}
fn submit_to_glt<T: ?Sized>(sub: T) {
  // …
}
  \end{minted}

  \href{https://github.com/rust-lang/rfcs/blob/master/text/0490-dst-syntax.md}{RFC 0490 DST syntax}. Only works for the \code{Sized} trait.

  Generics must be sized per default (its size is known at compile-time).
  But in some contexts it is allowed to be \enquote{maybe sized} (sized or unsized).
\end{frame}

\begin{frame}[fragile]{traits and trait bounds}
  \begin{minted}{rust}
fn compare(sub1: impl Submission,
           sub2: impl Submission) -> i8 {
  if sub1.mentions_rust() &&
    !sub2.mentions_rust() {
    -1
  } else if !sub1.mentions_rust() &&
             sub2.mentions_rust() {
    1
  } else {
    0
  }
}
  \end{minted}
\end{frame}

\begin{frame}[fragile]{traits and trait bounds}
  \begin{minted}{rust}
fn compare<T: Submission>
          (sub1: T, sub2: T) -> i8 {
  if sub1.mentions_rust() &&
    !sub2.mentions_rust() {
    -1
  } else if !sub1.mentions_rust() &&
             sub2.mentions_rust() {
    1
  } else {
    0
  }
}
  \end{minted}
\end{frame}

\begin{frame}[fragile]{traits and trait bounds}
  \textbf{Question:} Are these trait bounds the same?

  \pause
  \begin{itemize}
    \item No.
    \item \mintinline{rust}{impl Submission} requires two implementations of \code{Submission}
    \item \mintinline{rust}{<T: Submission>} requires two times the \emph{same} implementations of \code{Submission}
  \end{itemize}
\end{frame}

\begin{frame}[fragile]{traits and trait bounds}
  \begin{minted}{rust}
struct Talk {
    desc: String,
    duration: u16,
}

struct Workshop {
    desc: String,
    duration: u16,
}

impl Submission for Talk { /* … */ }
impl Submission for Workshop { /* … */ }
  \end{minted}
\end{frame}

\begin{frame}[fragile]{traits and trait bounds}
  \begin{minted}{rust}
compare(
  Talk{
    desc: String::from("This talk …"),
    duration: 45
  }, Workshop{
    desc: String::from("In this workshop …"),
    duration: 90
  }
);
  \end{minted}

  \begin{minted}{text}
error[E0308]: mismatched types
[…]
expected struct `Talk`, found struct `Workshop`
  \end{minted}
\end{frame}

\begin{frame}[fragile]{traits and trait bounds}
  \begin{minted}{rust}
fn compare<T1: Submission, T2: Submission>
          (sub1: T1, sub2: T2) -> i8 {
  if sub1.mentions_rust() &&
    !sub2.mentions_rust() {
    -1
  } else if !sub1.mentions_rust() &&
             sub2.mentions_rust() {
    1
  } else {
    0
  }
}
  \end{minted}
  This covers the \code{(Talk, Workshop)} case, of course.
\end{frame}


\begin{frame}[fragile]{traits and trait bounds}
  \begin{itemize}
    \item A \emph{marker trait} is a trait requiring implementation of 0 methods. Used to \emph{mark} a property. See \href{https://doc.rust-lang.org/std/marker/trait.Send.html}{Send} and \href{https://doc.rust-lang.org/std/marker/trait.Sync.html}{Sync} traits.
    \item We can also conditionally implement a trait for any type that implements another trait. Implementations of a trait on any type that satisfies the trait bounds are called \emph{blanket implementations} and are extensively used in the Rust standard library.
    \item \emph{Auto traits} are \href{https://stackoverflow.com/a/49710956}{opt-in, built-in traits (OIBIT)} which are unstable and not covered here (\enquote{automatically implemented traits}). \enquote{Auto trait implementation} occur in the standard library.
  \end{itemize}
\end{frame}

\begin{frame}[fragile]{standard library traits}
  \begin{description}
    \item[Iterator trait] discussed before.
    \item[Add trait] implements operator overloading of the addition operator \mintinline{rust}{+}.
    \item[Deref trait] is used for immutable dereferencing operations, like \mintinline{rust}{*val}. \mintinline{rust}{fn deref(&self) -> &Self::Target} is required.
    \item[Drop trait] is used to run some code when a value goes out of scope. This is sometimes called a \emph{destructor}. \mintinline{rust}{fn drop(&mut self)} is required.
  \end{description}
\end{frame}

\begin{frame}[fragile]{standard library traits}
  \begin{description}
    \item[Send trait] is a marker trait for types that can be transferred across thread boundaries.
    \item[Sync trait] is a marker trait for types where it is safe to share references between threads.
    \item[Display trait] format trait for an empty format; \mintinline{rust}{{}}.
    \item[Debug trait] provides the output in a programmer-facing, debugging context; \mintinline{rust}{{:?}}.
  \end{description}
  Both require:
  \begin{minted}{rust}
fmt(&self, f: &mut Formatter)
   -> Result<(), Error>
  \end{minted}
\end{frame}


% POSTPONED “use” trait and pub
%   TODO sometimes it is necessary to “use” just the trait

% POSTPONED
% TODO associated type
%   https://stackoverflow.com/a/54803174
%   hetseq len.rs gives an example with colon(!) with "type Length: Num" inside its trait Length.
%   Associated types can be defined (e.g. type Item;) inside a trait (e.g. std::iter::Iterator)
%   indexing trait requires return type reference. But Rc is also a reference? One defined Output type.

% POSTPONED
% TODO trait objects as run time concept, type erasure
%    rust supports static and dynamic dispatching.
%    https://doc.rust-lang.org/1.30.0/book/first-edition/trait-objects.html
%    https://doc.rust-lang.org/book/ch17-02-trait-objects.html
%    `fn use(&Iterator)` illustrates dynamic dispatch with a vtable. Supports heterogeneity.
%    `dyn/trait objects <https://doc.rust-lang.org/nightly/edition-guide/rust-2018/trait-system/more-container-types-support-trait-objects.html>`_
%    `Rust notes: dynamic and global static objects <https://gist.github.com/DmitrySoshnikov/8439eac0a09d9fafe55a83c88d049117>`_
%    the Any type https://doc.rust-lang.org/nightly/core/any/trait.Any.html

% POSTPONED
%   TODO What is part of a trait signature w.r.t. to static & dynamic dispatch?

% POSTPONED
% TODO `What are Rust's exact auto-dereferencing rules? <https://stackoverflow.com/a/28552082>`_

\begin{frame}[fragile]{traits}
  \textbf{Question:} Can we provide references to traits?

  Not discussed today. Topic \emph{trait objects} and \emph{dynamic dispatch}.

  % TODO Does it matter whether we are generic over T or &T?
  %     https://stackoverflow.com/questions/28005134/how-do-i-implement-the-add-trait-for-a-reference-to-a-struct
\end{frame}

\section{Dialogue: generics}

% POSTPONED
% `Learning Generics in Rust <https://tutorialedge.net/rust/learning-generics-in-rust/>`_

\begin{frame}[fragile]{type systems}
  How can we maintain the DRY (Don't repeat yourself) principle in programming? \\[8pt]

  Opinionated answer:
  \begin{description}
    \item[strongly typed language] Generics
    \item[dynamically typed language] Duck typing
  \end{description}
\end{frame}

\begin{frame}[fragile]{generics introduction}
  \begin{minted}{rust}
use std::mem::{size_of,align_of};
fn main() {
  println!("{} ⇒ {}", size_of::<i32>(),
           align_of::<i32>());       // 4 ⇒ 4
  println!("{} ⇒ {}", size_of::<u8>(),
           align_of::<u8>());        // 1 ⇒ 1
  println!("{} ⇒ {}", size_of::<*mut i32>(),
           align_of::<*mut i32>());  // 8 ⇒ 8
  println!("{} ⇒ {}", size_of::<usize>(),
           align_of::<usize>());     // 8 ⇒ 8
  println!("{} ⇒ {}", size_of::<char>(),
           align_of::<char>());      // 4 ⇒ 4
}
  \end{minted}

  A more compact version would be nice.
\end{frame}

\begin{frame}[fragile]{generics introduction}
  \begin{minted}{rust}
  println!("{} ⇒ {}", size_of::<i32>(),
           align_of::<i32>());       // 4 ⇒ 4
  \end{minted}

  \textbf{Goal:} make \mintinline{rust}{i32} parametrizable. This example does \emph{not} compile:

  \begin{minted}{rust}
use std::mem::{size_of,align_of};
fn show(T) {
  println!("{} ⇒ {}", size_of::<T>(),
           align_of::<T>());
}
fn main() {
  show::<i32>();
  show::<u8>();
  // …
}
  \end{minted}
\end{frame}

\begin{frame}[fragile]{generics introduction}
  \begin{minted}{rust}
use std::mem::{size_of,align_of};
fn show<T>() {
  println!("{} ⇒ {}",
    size_of::<T>(),
    align_of::<T>());
}
fn main() {
  show::<i32>();      // 4 ⇒ 4
  show::<u8>();       // 1 ⇒ 1
  // …
}
  \end{minted}

  This example compiles.
\end{frame}


\begin{frame}[fragile]{generics introduction}
  
  {\Large We are generic over \emph{types}, not \emph{values}!} \\[5pt]

  So we cannot define a vector with \texttt{N} elements, where the vector is generic over \texttt{N} (length is value, not type).
\end{frame}

\begin{frame}[fragile]{generics and monomorphization}
  \begin{minted}{rust}
fn first<A, B>(tuple: (A, B)) -> A {
  let (a, b) = tuple;
  return a;
}

fn main() {
  assert_eq!(1, first((1, 9)));
}
  \end{minted}  
\end{frame}

\begin{frame}[fragile]{generics and monomorphization}
  How are generics implemented?
  Generics are implemented using \emph{monomorphization}.
  Monomorphization means specialized versions are generated for each type.

  \begin{minted}{rust}
assert_eq!(1, first((1, 9)));
// version for integers is generated
assert_eq!('l', first(('l', 'p')));
// version for characters is generated
assert_eq!(3.14159, first((3.14159, 2.718281)));
// version for floats is generated
  \end{minted}

  Monomorphization is how C++ templates are implemented (unlike Java interfaces).
\end{frame}

\begin{frame}[fragile]{generics and monomorphization}
  \begin{minted}{rust}
fn main() {
    let mut i = (1, 9);
    let mut c = ('l', 'p');
    let mut f = (3.14159, 2.718281);

    // avoids `first` to be inlined
    i.1 += 99;
    c.1 = 'r';
    f.1 = 42.0;

    assert_eq!(1, first(i));
    assert_eq!('l', first(c));
    assert_eq!(3.14159, first(f));
}
  \end{minted}  
\end{frame}

\begin{frame}[fragile]{generics and monomorphization}
  \textbf{Advantage:} fast, specialized/optimized versions \\
  \textbf{Disadvantage:} executable size increases \\[5pt]

  Only the first \mintinline{rust}{first} call: 2,770,592 bytes \\
  With two \mintinline{rust}{first} calls: 2,772,032 bytes \\
  With three \mintinline{rust}{first} calls: 2,804,176 bytes \\[5pt]

  Differences are +1440 and +32144 bytes. Cannot tell why it is a non-linear increase, but the point is that the executable size increases.
\end{frame}

% POSTPONED
% TODO monomorphisation and executable size
%    `What is monomorphisation with context to C++? <https://stackoverflow.com/q/14189604>`_
%    `fn use<T: Iterator>(t: T)` illustrates static dispatch (generic/parametric polymorphism). For each different choice of T, a custom compilation of use will be generated (enables optimization) (increases binary size) (like templates in C++).

% POSTPONED
% TODO generics
% unit structs: struct Nil;   ⇒ instantiation always gives the same element? definition = instantiation? ⇒ Peano axioms
%   `struct A;`    versus   `struct A {};`



% TODO generic arrays
% TODO dyn keyword ⇒ https://stackoverflow.com/a/50987296

\begin{comment}
% POSTPONED
struct Unit {}
trait Next {}

//impl Next for Unit {}

fn count(u: impl Next) -> u64 {
    match u {
        Next(v) => count(v) + 1,
        Unit{} => 0,
    }
}

fn main() {
    let u = Unit {};
    let t = Unit {};

    println!("{:p}", &t);
    println!("{:p}", &u);
    
    println!("{}", count(Unit{}));
}
\end{comment}


% POSTPONED
% TODO there is no kind of dispatching, right? So no method name occurs twice and based on the traits, we dispatch
%    `Wikipedia: Dependent type <https://en.wikipedia.org/wiki/Dependent_type>`_  not possible
% TODO which objects can be generic? functions? types?


% TODO newtype idiom
%    https://doc.rust-lang.org/rust-by-example/generics/new_types.html

% `HackerNews: Phantom Types in Rust <https://news.ycombinator.com/item?id=6217314>`_ ⇒ `Phantom types in rust <https://bluishcoder.co.nz/2013/08/15/phantom_types_in_rust.html>`_
%    https://doc.rust-lang.org/rust-by-example/generics/phantom.html
%    Quiz: \item[Phantom types are] ~\uncover<8->{generic types that don't use all type parameters to allow distinct types with the same (effective) types}

% TODO typestate
%   `The Typestate Pattern in Rust <http://cliffle.com/blog/rust-typestate/>`_
%   `Typestate analysis <https://en.wikipedia.org/wiki/Typestate_analysis>`_
%   `Typestate Is Dead, Long Live Typestate! <https://pcwalton.github.io/2012/12/26/typestate-is-dead.html>`_
%   TODO find application in stdlib

\section{Dialogue: Algebraic data types}

\begin{frame}[fragile]{algebraic data types}
  \begin{quote}
    In computer programming, especially functional programming and type theory, an algebraic data type (ADT) is a kind of composite type, i.e., a type formed by combining other types. \\
    ---\href{https://en.wikipedia.org/wiki/Algebraic_data_type}{Wikipedia}
  \end{quote}
\end{frame}

\begin{frame}[fragile]{algebraic data types}
  \begin{minted}{rust}
use std::fmt;

enum List {
    Nil,
    Cons(Box<List>, u64),
}
  \end{minted}
  Algebraic? Sums and products. \code{List} is the sum of \code{Nil} and \code{Cons(\_)}. \code{Cons} is the product of \code{Box$<$List$>$} and \code{u64}.
  Boxing? Avoids \mintinline{text}{recursive type `List` has infinite size}.

  Article: \href{https://blog.softwaremill.com/algebraic-data-types-in-four-languages-858788043d4e}{Algebraic data types in four languages} (namely Haskell, Scala, rust, and TypeScript)
\end{frame}

\begin{frame}[fragile]{algebraic data types}
  \begin{minted}{rust}
impl fmt::Display for List {
  fn fmt(&self, f: &mut fmt::Formatter<'_>)
     -> fmt::Result {
    match self {
      List::Cons(inner, item)
        => write!(f, "(cons {} {})", item, inner),
      List::Nil
        => write!(f, "nil"),
    }
  }
}
  \end{minted}
  Recognize that \code{Cons} is addressed by \mintinline{rust}{List::Cons}.
\end{frame}

\begin{frame}[fragile]{algebraic data types}
  \textbf{Call:}
  \begin{minted}{rust}
fn main() {
  println!("{}", List::Cons(Box::new(
    List::Cons(Box::new(List::Nil), 1)), 2));
}
  \end{minted}
  \textbf{Output:}
  \begin{minted}{text}
(cons 2 (cons 1 nil))
  \end{minted}
\end{frame}


\begin{frame}[fragile]{algebraic data types}
  \begin{minted}{rust}
use std::fmt;

enum Tree {
    Node(Box<Tree>, Box<Tree>),
    Leaf(u32),
    Empty,
}
  \end{minted}
\end{frame}

\begin{frame}[fragile]{algebraic data types}
  \begin{minted}{rust}
impl fmt::Display for Tree {
  fn fmt(&self, f: &mut fmt::Formatter<'_>)
     -> fmt::Result {
    match self {
      Tree::Node(left, right)
        => write!(f, "({}, {})", left, right),
      Tree::Leaf(item)
        => write!(f, "({})", item),
      Tree::Empty
        => write!(f, "_"),
    }
  }
}
  \end{minted}
\end{frame}

\begin{frame}[fragile]{algebraic data types}
  \textbf{Call:}
  \begin{minted}{rust}
fn main() {
    println!("{}", Tree::Node(
        Box::new(Tree::Leaf(42)),
        Box::new(Tree::Empty)
    ));
}
  \end{minted}

  \textbf{Output:}
  \begin{minted}{text}
((42), _)
  \end{minted}
\end{frame}

\section{Dialogue: Type inference}

\begin{frame}[fragile]{type inference}
  \begin{minted}{rust}
fn main() {
    let a = vec![135, 145, 4, 0];
    println!("{}", a[0] + 4f64);
}
  \end{minted}
\end{frame}

\begin{frame}[fragile]{type inference}
  \begin{minted}[fontsize=\scriptsize]{text}
error[E0277]: cannot add `f64` to `{integer}`
 --> src/main.rs:3:25
  |
3 |  println!("{}", a[0] + 4f64);
  |                      ^ no implementation for `{integer} + f64`
  |
  = help: the trait `std::ops::Add<f64>` is
          not implemented for `{integer}`

error: aborting due to previous error

For more information about this error, try `rustc --explain E0277`.
error: could not compile `playground`.

To learn more, run the command again with --verbose.
  \end{minted}
  The rust compiler keeps data types abstract as long as possible. The bit size of \code{\{integer\}} is not defined.
\end{frame}

\begin{frame}[fragile]{type inference}
  This enables \emph{type inference}. The type of a value is \emph{inferred} by the context it is used in.
  \begin{minted}{rust}
fn main() {
  let a = 3;
  println!("{}", a + 4u32);
}
  \end{minted}
  The type \mintinline{rust}{u32} for \code{a} is inferred.
\end{frame}

\begin{frame}[fragile]{type inference}
  \begin{minted}{rust}
fn main() {
  let a = -3;
  println!("{}", a + 4u32);
}
  \end{minted}
  Gives an error, because there is no solution for type inference.
  \begin{minted}{text}
error[E0277]: the trait bound
  `u32: std::ops::Neg` is not satisfied
  \end{minted}
\end{frame}

% TODO `Rc<RefCell<_>>` shows that partial application of type inference is also possible (via `docs: cell <https://doc.rust-lang.org/std/cell/index.html>`_)

\begin{frame}[fragile]{type inference}
  I struggled with a type problem once. Reproduction of the problem didn't show a problem, because type inference kicked in. Reduced example:

  \begin{minted}{rust}
fn main() {
  let l = 5;
  println!("{} ⇒ {}", 'c',
    &format!("{:016b}", 0b01011)[16-l..16]);
}
  \end{minted}

  Does it compile?
\end{frame}

\begin{frame}[fragile]{type inference}
  Yes, \textbf{output:}
  \begin{minted}{text}
c ⇒ 01011
  \end{minted}
  \vspace{10pt}

  BTW, out of bounds errors are either compile-time errors (few cases) or run-time errors (all other cases). Same for debug and release mode.
\end{frame}

\begin{frame}[fragile]{type inference}
  \begin{minted}{rust}
fn main() {
  let l = 5u8;
  println!("{} ⇒ {}", 'c',
    &format!("{:016b}", 0b01011)[16-l..16]);
}
  \end{minted}

  Recognize that \code{l} is enforced with type \mintinline{rust}{5u8}.

  Does it compile?
\end{frame}

\begin{frame}[fragile]{type inference}
  No, \textbf{output:}
  \begin{minted}{text}
the type `std::string::String` cannot
be indexed by `std::ops::Range<u8>`
  \end{minted}

  The first example inferred type \mintinline{rust}{usize} for \code{l}.
  The second example uses \mintinline{rust}{u8}. Indexing requires \mintinline{rust}{usize}.
\end{frame}


\section{Dialogue: Type aliasing}

\begin{frame}[fragile]{type aliasing}
  \begin{minted}{rust}
type Point = (u8, u8);

fn main() {
  let p: Point = (41, 68);
  assert_eq!(p, (41, 68));
}
  \end{minted}
  \mintinline{rust}{Point} is a type alias of \mintinline{rust}{(u8, u8)}.
\end{frame}

\begin{frame}[fragile]{type aliasing}
  \begin{minted}{rust}
type Point = (u8, u8);
type Two = (u8, u8);

fn main() {
  let p: Point = (41, 68);
  let t: Two = (41, 68);
  assert_eq!(p, (41, 68));
  assert_eq!(p, t);
}
  \end{minted}

  \emph{Off-Topic reminder:} \mintinline{rust}{p.0} accesses the first \mintinline{rust}{u8}, i.e. 41 (other languages use [] syntactically for indexing).
\end{frame}

\begin{frame}[fragile]{type aliasing: edge cases}
  \begin{minted}{rust}
struct A {
  a: u8,
  b: char,
}
type X = A;
impl X {
  fn inc(&mut self) {
    self.a += 1;
  }
}
fn main() {
  let mut inst = A {a: 42, b: 'l'};
  inst.inc();
  println!("{}", inst.a); // 43
}
  \end{minted}
\end{frame}

\begin{frame}[fragile]{type aliasing: edge cases}
  \begin{minted}{rust}
type ExtendedUint = u32;

impl ExtendedUint {
    fn inc(&self) -> u32 {
        self.0 + 1
    }
}

fn main() {
    let h: ExtendedUint = 41;
    println!("{:?}", h.inc());
}
  \end{minted}

  Does it compile?
\end{frame}

\begin{frame}[fragile]{type aliasing: edge cases}
  \begin{minted}[fontsize=\scriptsize]{text}
error[E0390]: only a single inherent implementation
              marked with `#[lang = "u32"]` is allowed
              for the `u32` primitive
 --> src/main.rs:3:1
  |
3 | / impl ExtendedUint {
4 | |     fn inc(&self) -> u32 {
5 | |         self.0 + 1
6 | |     }
7 | | }
  | |_^
  |
help: consider using a trait to implement these methods
 --> src/main.rs:3:1
  |
3 | / impl ExtendedUint {
4 | |     fn inc(&self) -> u32 {
5 | |         self.0 + 1
6 | |     }
7 | | }
  | |_^
error: aborting due to previous error
  \end{minted}
\end{frame}

% TODO
\begin{comment}
\section{Dialogue: Referential transparency}

% REMARK this belongs to the memory managment section, not to the type system.

* rust violates referential transparency: sometimes assignment makes the program compile::

    // replace s with its definition ⇒ crashes
    // this ⇒ compiles.

    #![allow(unused)]
    fn main() {
    let s = String::from(" Mary   had\ta\u{2009}little  \n\t lamb");
    let mut iter = s.split_whitespace();
    assert_eq!(Some("Mary"), iter.next());
    assert_eq!(Some("had"), iter.next());
    assert_eq!(Some("a"), iter.next());
    assert_eq!(Some("little"), iter.next());
    assert_eq!(Some("lamb"), iter.next());

    assert_eq!(None, iter.next());
    }
\end{comment}

\section{Dialogue: type systems}

% REMARK Substructural type system
%   `Wikipedia: Substructural type system <https://en.wikipedia.org/wiki/Substructural_type_system>`_
%   e.g. “type X must only be used at most once ⇒ x + x is not allowed”
%   too specific to be discussed

\begin{frame}[standout]
  Haskell
\end{frame}

\begin{frame}[fragile]{Haskell type classes}
  Rust's traits are based on Haskell's type classes.

  \begin{quote}
    Type classes are defined by specifying a set of function or constant names, together with their respective types, that must exist for every type that belongs to the class. \\
    ---\href{https://en.wikipedia.org/wiki/Type_class}{Wikipedia: Type class}
  \end{quote}
\end{frame}

\begin{frame}[fragile]{Haskell type class example}
  \begin{minted}{haskell}
class Eq a where
  (==) :: a -> a -> Bool
  (/=) :: a -> a -> Bool
  \end{minted}
\end{frame}

\begin{frame}[fragile]{Haskell: default behavior}
  \begin{minted}{haskell}
class Monoid a where
    mempty  :: a
    mappend :: a -> a -> a

    -- | Fold a list using the monoid.
    --
    -- For most types, the default definition
    -- for 'mconcat' will be used, but the
    -- function is included in the class
    -- definition so that an optimized version
    -- can be provided for specific types.
    mconcat :: [a] -> a
    mconcat = foldr mappend mempty
  \end{minted}
  \source{\href{https://hackage.haskell.org/package/base-4.12.0.0/docs/src/GHC.Base.html\#Monoid}{Haskell base: Monoid}}
\end{frame}

\begin{frame}[fragile]{Haskell: minimal complete definition}
  In Haskell documentation, you often find a \emph{minimal complete definition} list:

  \vspace{20pt}
  \begin{quote}
    \textbf{Minimal complete definition} \\
    foldMap | foldr
  \end{quote}

  In order to implement \href{https://hackage.haskell.org/package/base-4.12.0.0/docs/Data-Foldable.html}{Data.Foldable}, you must implement at least \code{foldMap} and \code{foldr}. Otherwise the default implementations might not work.

  I have never seen such documented requirements yet in rust.
\end{frame}

\begin{frame}[fragile]{Haskell: differences}
  What is different from rust's traits?
  \begin{enumerate}
    \item Haskell does not bind the implementation to some object (i.e. \mintinline{rust}{&self})
    \item Initially rust did not have support for higher-order typeclasses. Now it has. Haskell always had.
  \end{enumerate}

  based on \href{https://stackoverflow.com/a/32659986}{stackoverflow}
\end{frame}

\begin{frame}[fragile]{Haskell: algebraic data types}
  First example:
  \begin{minted}{haskell}
data List a = Nil | Cons a (List a)
  \end{minted}

  Second example:
  \begin{minted}{haskell}
data Tree = Empty
          | Leaf Int
          | Node Tree Tree
  \end{minted}
\end{frame}

\begin{frame}[fragile]{Haskell: algebraic data types}
  Pattern matching:
  \begin{minted}{haskell}
hasChild :: Tree -> Bool
hasChild Empty      = False
hasChild (Leaf _)   = False
hasChild (Node _ _) = True
  \end{minted}
\end{frame}

\begin{frame}[standout]
  C++
\end{frame}

\begin{frame}[fragile]{C++: templates and concepts}
  \begin{itemize}
    \item C++ extends C with classes (compare with \href{https://www.youtube.com/watch?v=86xWVb4XIyE}{The Essence of C++})
    \item Interfaces somewhat correspond to abstract classes % via https://en.wikipedia.org/wiki/Protocol_(object-oriented_programming)
    \item Templates were introduced at about 1995; allow code generation and metaprogramming; unrestricted
    \item Strong notion of subtyping, polymorphism and data encapsulation
    \item Lots of inspiration for rust
    \item C++20 introduces concepts (first discussed in 1998)
  \end{itemize}
\end{frame}

\begin{frame}[fragile]{C++: type inference}
  \begin{minted}{c++}
#include <bits/stdc++.h> 
using namespace std; 
int main() 
{ 
    auto x = 4; 
    auto y = 3.37; 
    auto ptr = &x; 
    cout << typeid(x).name() << endl    // i
         << typeid(y).name() << endl    // d
         << typeid(ptr).name() << endl; // Pi
    return 0; 
} 
  \end{minted}

  \code{auto} keyword since C++11.

  \source{\href{https://www.geeksforgeeks.org/type-inference-in-c-auto-and-decltype/}{geeksforgeeks.org}}
\end{frame}

\begin{frame}[fragile]{C++: default behavior through subtyping}
  \begin{minted}{c++}
struct A {
  int x;
  // user-defined default constructor
  A(int x = 1): x(x) {}
};
 
struct B: A {
  // B::B() is inherited from A::A()
};
  \end{minted}

  \source{\href{https://en.cppreference.com/w/cpp/language/default_constructor}{C++: default\_constructor}}
\end{frame}

\begin{frame}[fragile]{C++: abstract class}
  \begin{minted}{c++}
class Box {
  public:
    virtual double getVolume() = 0;
      
  private:
    double length;
    double breadth;
    double height;
};
  \end{minted}
  A class becomes abstract, because one of its methods is abstract (unimplemented).

  \source{\href{https://www.tutorialspoint.com/cplusplus/cpp_interfaces.htm}{tutorialspoint.com}}
\end{frame}

\begin{frame}[fragile]{C++: templates}
  \begin{minted}{c++}
template <class T>
T GetMax (T a, T b) {
  T result;
  result = (a>b)? a : b;
  return (result);
}

// …
GetMax(myAge, yourAge) // integers
GetMax(lakeTemperature, seaTemperature) // float
  \end{minted}
  \source{\href{http://www.cplusplus.com/doc/oldtutorial/templates/}{cplusplus.com}}
\end{frame}

\begin{frame}[fragile]{C++: templates}
  \begin{minted}{c++}
template <unsigned int n>
struct factorial {
  enum { value = n * factorial<n - 1>::value };
};

template <>
struct factorial<0> {
  enum { value = 1 };
};
  \end{minted}

  \source{\href{https://citeseerx.ist.psu.edu/viewdoc/summary?doi=10.1.1.14.3670}{C++ templates are Turing-complete}}
\end{frame}

\begin{frame}[fragile]{C++: concepts}
  \begin{minted}{c++}
template<typename T>
concept EqualityComparable = requires(T a, T b) {
    { a == b } -> std::boolean;
    { a != b } -> std::boolean;
};
  \end{minted}

  $\rightarrow$ via \href{https://en.wikipedia.org/wiki/Concepts_(C%2B%2B)}{Wikipedia: Concepts (C++)}
\end{frame}

\begin{frame}[standout]
  Java
\end{frame}

\begin{frame}[fragile]{Java and interfaces}
  \begin{minted}{java}
interface Bicycle {
    void speedUp(int increment);
}

class ACMEBicycle implements Bicycle {
    int speed = 0;
    void speedUp(int increment) {
         speed = speed + increment;   
    }
}
  \end{minted}

  \source{\href{https://docs.oracle.com/javase/tutorial/java/concepts/interface.html}{docs.oracle.com}}
\end{frame}

\begin{frame}[fragile]{Java and interfaces}
  \begin{itemize}
    \item \mintinline{java}{Type implements Interface} declare implementation of an interface
    \item list of methods and constants required to satisfy this interface
    \item emulates multiple inheritance, embedded in subtyping hierarchy
    \item class can implement some interface incompletely ($\rightarrow$ abstract class)
    \item \mintinline{java}{Interface extends Interface1, Interface2} defines inheritance for interfaces
    \item Java 8: default and static methods may have implementation in interface definition
    \item Java 9: private and private static methods
  \end{itemize}
\end{frame}

\begin{frame}[standout]
  Go
\end{frame}

\begin{frame}[fragile]{Go: interfaces and contracts}
  \begin{minted}{go}
package main

import (
  "fmt"
)

type Adder interface {
  add() uint32
}
  \end{minted}
\end{frame}

\begin{frame}[fragile]{Go: interfaces and contracts}
  \begin{minted}{go}
type Add struct {
  x uint32
  y uint32
}

func (a Add) add() uint32 {
  return a.x + a.y
}

func main() {
  var sum Adder = Add{10, 32}
  fmt.Println(sum.add())
}
  \end{minted}
\end{frame}

\begin{frame}[fragile]{Go: interfaces and contracts}
  \begin{itemize}
    \item Go has interfaces (no constant types, no subtyping, but embedding)
    \item Go types implement an interface if they implement the corresponding method signatures implicitly (no \code{implements} keyword)
    \item Go does not have generics (see \href{https://golang.org/pkg/sort/}{sort}, code duplication and casting is common)
    \item Lack of generics was identified as one of the Top 3 major challenges in the \href{https://blog.golang.org/survey2018-results}{2018 developer survey}
  \end{itemize}
\end{frame}

\begin{frame}[fragile]{Go: interfaces and contracts}
  A draft for Go 2.x:

  \begin{minted}{go}
func Print(s []T) {
  for _, v := range s {
    // cannot coerce to string implicitly.
    // must implement String() → no guarantee.
    fmt.Println(v)
  }
}
  \end{minted}

  \source{\href{https://github.com/golang/proposal/blob/master/design/go2draft-contracts.md}{go2draft contracts on github}}
\end{frame}

\begin{frame}[fragile]{Go: interfaces and contracts}
  A draft for Go 2.x:

  \begin{minted}{go}
contract stringer(T) {
  T String() string
}

func Stringify(type T stringer)(s []T)
              (ret []string)
{
  for _, v := range s {
    ret = append(ret, v.String())
  }
  return ret
}
  \end{minted}
\end{frame}

\begin{frame}[fragile]{Go: interfaces and contracts}
  \begin{itemize}
    \item Draft is 6 months old
    \item Current 1.14 release (yesterday) does not feature contracts
    \item Go contracts are based on C++ concepts
    \item Summary: \href{https://github.com/luciotato/golang-notes/blob/master/OOP.md}{Golang concepts from an OOP point of view}
  \end{itemize}
\end{frame}

\begin{frame}[fragile]{Nominal type system versus structural type system}
  \href{https://en.wikipedia.org/wiki/Nominal_type_system}{Nominal type system}:
  \begin{itemize}
    \item \enquote{compatibility and equivalence of data types is determined by explicit declarations and/or the name of the types}
    \item \code{Who implements What}, name is normative
  \end{itemize}

  \href{https://en.wikipedia.org/wiki/Structural_type_system}{Structural type system}:
  \begin{itemize}
    \item \enquote{type compatibility and equivalence are determined by the type's actual structure or definition}
    \item If \code{Who} has the structure of \code{What}, \code{Who implements What}
  \end{itemize}

  \emph{My point:} Go contracts / C++ concepts are structural features of their respective type system
\end{frame}

\section{Epilogue}

\begin{frame}[fragile]{Opinionated summary}
  Rust is multiparadigmatic.
  \begin{description}
    \item[imperative] all features except for goto 😋
    \item[OOP] no subtyping provided, but default implementations for code reuse. No polymorphism, but generics. Data encapsulation via \code{pub}.
    \item[FP] Iterator trait features many basic function. Higher order functions via generics, Fn/FnOnce/FnMut traits, and closures. But no dependent types or alike.
    \item[decl./DSL] not really, but macros allow some flexibility and operator overloading through traits
  \end{description}
  % REMARK appropriate joke at this topic: it does not feature return-oriented programming
\end{frame}

\begin{frame}[fragile]{Quiz - part 1}
  \begin{description}
    \item[rust traits are based on] ~\uncover<2->{Haskell's type classes}
    \item[\mintinline{Java}{Type implements Traits} is written as] ~\uncover<3->{\mintinline{rust}{impl Trait for Type}}
    \item[traits can be implemented for] ~\uncover<4->{\mintinline{rust}{struct}, \mintinline{rust}{enum}, \mintinline{rust}{union}}
    \item[generics parametrize something with] ~\uncover<5->{some type}
    \item[the orphan rule states] ~\uncover<6->{a child (implementation) must know its parents (trait, struct) [to implement a trait]}
    \item[monomorphisation means] ~\uncover<7->{the source code is generated for each type that is used, just like C++ templates}
  \end{description}
\end{frame}

\begin{frame}[fragile]{Quiz - part 2}
  \begin{description}
    \item[Originally, C++ designed templates to be] ~\uncover<2->{unrestricted}
    \item[C++ templates are] ~\uncover<3->{turing-complete}
    \item[C++20 allows compile-time protocols with] ~\uncover<4->{concepts}
    \item[A difference between Rust traits and Java interfaces is] ~\uncover<5->{incomplete implementation}
    \item[Go interfaces are not nominal, but] ~\uncover<6->{structural}
    \item[Algebraic data types are] ~\uncover<7->{product and sum of sets of data types}
    \item[Marker traits are] ~\uncover<8->{traits that don't require any methods to be implemented}
  \end{description}
\end{frame}

% POSTPONED
%If traits \& generics finished:
%
%  EITHER Data structures and lifetimes
%  OR Modularization
%  OR More types AND/OR Debugging, i18n and dates

\begin{frame}[fragile]{Next time}
  \begin{description}
    \item[Next meetup] Wed, 2020/03/25.
    \item[March] ? (mostly depends on David's content) \\
      My next topics:
      \begin{itemize}
        \item trait objects, dyn keyword
        \item typestate pattern, newtype idiom
        \item modularization in crates, and modules. pub keyword
      \end{itemize}
    \item[April] Hacker Jeopardy
  \end{description}
\end{frame}

\begin{frame}[standout]
  Thank you!

  \includegraphics[width=40pt]{images/rustacean-flat-happy.png}
\end{frame}

\end{document}
